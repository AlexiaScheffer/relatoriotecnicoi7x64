\documentclass[12pt,a4paper,brazilian,utf8]{ppgsi}

\title{Título do Relatório Técnico}

% Autores, como aparecem na capa do RT
\coverauthor{Aléxia Carolina Scheffer da Silva\\ Giovanni Bergaminni\\ Gustavo Jyun Hayashida\\ Matheus Barbosa\\ Norton Trevisan Roman}

% Autores do documento
\author{Aléxia Carolina Scheffer da Silva\inst{1}\\ Giovanni Bergaminni\inst{1}\\ Gustavo Jyun Hayashida\inst{1}\\ Matheus Barbosa\inst{1}\\ Norton Trevisan Roman\inst{1}}

\address{Escola de Artes, Ciências e Humanidades -- Universidade de São Paulo\\
    São Paulo -- SP, Brasil
    \email{norton@usp.br}
}

% Número do RT
\numero{XXX/2019}

% Mês e ano do RT
\mes{11}
\ano{2019}

\begin{document}

\maketitle

\begin{abstract}
    Texto  resumo do nosso rt - ???. Última atualização: 17/11/2019.
\end{abstract}

\section{Avaliação do professor}
    \begin{itemize}
		\item ilustrações, diagramas, gráficos (comparativos tb)
        \item leitura clara e didática
		\item referência em tudo
        \item fontes confiavéis
            \begin{itemize}
        		\item livros didáticos
                \item livros científicos
                \item artigos científicos (revistas, congresso)
                \item relatórios técnicos oficiais
                \item working papers (relatório técnico de empresas)
                \item manuais dos fabricantes das arquiteturas (site oficial)
        	\end{itemize}
	\end{itemize}

\section{Introdução}
    (Texto da introdução - ???)
    \cite{Bourdieu1977}

\section{Histórico}
    (Texto do histórico - suas origens  e estado atual - Giovanni)

\section{Conceitos Básicos}
    (Texto sobre Conceitos Básicos - que vamos utilizar na pesquisa - ciclo único, pipeline, superpipeline, superescalar - Jyun e Matheus)
    
\section{Detalhes}
    (detalhes - registradores, frequência de clock, tipo de memória, cache, núcleos, threads, etc - Matheus)

\section{Pretensão}
    (Texto sobre a pretensão - geral ou específico, pq foi criada? - Alexia)

\section{Uso atual}
    (Texto sobre o uso atual - onde é usada, onde se encontra - Matheus)

\section{Conjunto de instruções (ISA)}
    (Texto sobre conjunto de instruções - instruções que fornece ao programador - detalhe das instruções, codificação dentro do datapath - ???)

\section{Microarquitetura}
    (Texto sobre microarquitetura - como as instruções do ISA são realmente implementadas no circuito - ???)

\section{Desempenho}
    (Texto sobre desempenho - mesmo benchmarks da outra arquitetura - Giovanni)

\section{Sobre a arquitetura para comparação}
    (Texto sobre a arquitetura para comparação - ???) 

\section{Comparação entre arquiteturas}
    (Texto sobre a comparação com outra arquitetura atual (mesmo benchmark) - estruturas ou funções exigem mais (ou menos) operações em cada arquitetura - vantagens de cada uma - Alexia)

\section{Conclusão}
    (Texto sobre conclusão - ???)

\section{Considerações Finais}
    (Texto sobre considerações finais - ???)

\section*{Agradecimentos}
    (Texto sobre agradecimentos pertinentes - gradecer ao outro grupo pela comparação, agradecer ao professor Norton, agradecer a EACH, agredecer a diretoria - ???)

\section*{Demais Autores}
    (Lista dos nomes, filiação e e-mail dos demais autores do documento, na forma de uma tabela.)
    
% Início da bibliografia
\bibliographystyle{ppgsi}
\bibliography{arquivo_da_bibliografia}

% Apêndices
\appendix

\appsection{Apêndice 1}
    Texto do apêndice 1

\appsection{Apêndice 2}
    Texto do apêndice 2

\end{document}
